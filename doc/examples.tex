\section{Introduction}
\label{SECTION:intro}

Note: Should have plots of raw data histograms superposed on the final restraints in all cases.

This Chapter describes the use of \MDT\ for updating many of the \MODELLER\ restraint libraries, including 
stereochemical, non-bonded, and homology-derived restraints:

\begin{enumerate}

\item Stereochemical restraints:

\begin{itemize}
\item chemical bonds: $p(\mbox{Bond}/\mbox{BondType})$;

\item chemical angles: $p(\mbox{Angle}/\mbox{AngleType})$;

\item improper dihedral angles as defined in the CHARMM residue topology file: $p(\mbox{Dihedral}/\mbox{DihedralType})$;

\item the $\omega$ dihedral angle of the peptide bond: $p(\omega/\mbox{ResidueType}_{+1})$, where 
      $\mbox{ResidueType}_{+1}$ refers to the residue type following the residue with the $\omega$ dihedral angle;

\item the $\Phi$ and $\Psi$ dihedral angles: $p(\Phi/\mbox{ResidueType})$, $p(\Psi/\mbox{ResidueType})$;

\item the sidechain dihedral angles: $p(\chi_1/\mbox{ResidueType})$, $p(\chi_2/\mbox{ResidueType})$, 
      $p(\chi_3/\mbox{ResidueType})$, and \\ $p(\chi_4/\mbox{ResidueType})$;

\item the mainchain conformation: $p(\Phi, \Psi/\mbox{ResidueType})$.
\end{itemize}

\item Non-bonded restraints:
\begin{itemize}
\item the mainchain hydrogen bonding restraints: $p(h/d,a)$;

\item the non-bonded pair of atom triplets: $p(d,\alpha_1,\alpha_2,\theta_1,\theta_2,\theta_3/t_1, t_2)$.
\end{itemize}

\item Homology-derived restraints:
\begin{itemize}
\item distance: $p(d/d')$.
\end{itemize}

\end{enumerate}

The following sections will outline the process of starting with the Protein Data Bank (PDB) and ending up with 
the \MODELLER\ restraint library files. We will describe the rationale for the process, input data sets, 
programs and scripts used, and even the analysis of the results. Many of the input and output files are not part 
of the \MDT\ distribution, but they should be available from the restraint-generation directory 
(\eg, {\tt $\tilde{}$sali/My/Work/Projects/constr2005/}).

The overall approach is to construct appropriately accurate, smooth, and transformed surfaces based
on the statistics in PDB for use as spatial restraints for model building. The restraints from the first 
iteration will be used to construct many models, which in turn will be used to re-derive the equivalent 
restraints from the models. These model-derived restraints will then be compared against the original 
PDB-derived restraints to find problems and get indications as to how to change the restraints so that 
models are statistically as similar to PDB structures as possible.

\section{Stereochemical restraints}

\subsection{The sample}

The starting point for deriving the restraints in this section consists of 9,365 chains 
that are representative of the 65,629 chains in the October 2005 version of PDB. The 
representative set was obtained by clustering all PDB chains with MODELLER-8, such that 
the representative chains are from 30 to 3000 residues in length and are sharing less than 
60\% sequence identity to each other (or are more than 30 residues different in length). This
is the corresponding \MODELLER\ script:

\Include{../constr2005/cluster-PDB/make-pdb60.py}

The actual chains for restraint derivation are in fact a subset of the 9,365 representative
chains, corresponding to the 4,532 crystallographic structures determined at least at 2 $\Ang$ 
resolution (the representative structure for each group is the highest resolution x-ray structure
in the group). This decision was made by looking at the distribution of the $\chi_1$ dihedral angles
as a function of resolution (Section on $\chi_1$) and the distribution of resolutions 
themselves for all 9,365 representative chains, using this MDT script:

\IncludeOut{../constr2005/impact-of-resolution/make-mdt.py}
           {../constr2005/impact-of-resolution/make-mdt.log}

\noindent which resulted in the MDT table {\tt mdt2.mdt} that was plotted with
the script:

\IncludeOut{../constr2005/impact-of-resolution/asgl.py}
           {../constr2005/impact-of-resolution/asgl2-a.pdf}
\IncludePDF{../constr2005/impact-of-resolution/asgl2-a.pdf}

\subsection{Chemical Bonds}

The \MDT\ table is constructed with the following \MDT\ Python script:

\IncludeOut{../constr2005/bonds/make-mdt.py}
           {../constr2005/bonds/mdt.mdt}

The contents of the \MDT\ table are then plotted with \ASGL\ as follows:

\IncludeOut{../constr2005/bonds/asgl.py}
           {../constr2005/bonds/asgl1-a.pdf}

\IncludePDF{../constr2005/bonds/asgl1-a.pdf}

Inspection of the plots shows that all distributions are mono-modal, but most are distinctly 
non-Gaussian. However, at this point, Gaussian restraints are still favored because the 
ranges are very narrow and because the non-Gaussian shape of the histograms
may results from the application of all the other restraints (this supposition will
be tested by deriving the corresponding distributions from the models, not PDB structures). Also, 
although many distributions are quite symmetrical, not all of them are. Therefore, there is the question
of how best to fit a restraint to the data. There are at least three possibilities, in principle: 
(i) calculating the average and standard deviation from all (subset) of the data, (ii) least-squares fitting of 
the Gaussian model to the data, and (iii) using cubic splines of the data. The first option was adopted here:
The mean and standard deviation will be the parameters of the analytically defined bond restraint for MODELLER.

And the final \MODELLER\ \MDT\ library is produced with:

\IncludeOut{../constr2005/bonds/modlib.py}
           {../constr2005/bonds/bonds.py}


\subsection{Chemical Angles}

As for the bonds above, the \MDT\ table is constructed with the following
\MDT\ Python script:

\IncludeOut{../constr2005/angles/make-mdt.py}
           {../constr2005/angles/mdt.mdt}

The contents of the \MDT\ table are then plotted with \ASGL\ as follows:

\IncludeOut{../constr2005/angles/asgl.py}
           {../constr2005/angles/asgl1-a.pdf}

\IncludePDF{../constr2005/angles/asgl1-a.pdf}

The situation is similar to that for the chemical bonds, except that there are also four cases of bi-modal
(as opposed to mono-modal) distributions: Asp:CB:CG:OD2, Asp:OD2:CG,OD1, Pro:CB:CG:CD, and Pro:CD:N:CA 
angles. The Asp bi-modal distribution may result from crystallographers mislabeling carboxyl oxygens for 
the protonated state of the sidechain (which is interesting because the corresponding angles might 
be used as a means to assign the protonation state). The mean values for these angles were edited by 
hand. Otherwise exactly the same considerations as for bonds apply here.

And the final \MODELLER\ \MDT\ library is produced with:

\IncludeOut{../constr2005/angles/modlib.py}
           {../constr2005/angles/angles.py}


\subsection{Improper Dihedral Angles}

Exactly the same situation applies as for the chemical bonds.

\IncludeOut{../constr2005/impropers/make-mdt.py}
           {../constr2005/impropers/mdt.mdt}
\IncludeOut{../constr2005/impropers/asgl.py}
           {../constr2005/impropers/asgl1-a.pdf}
\IncludeOut{../constr2005/impropers/modlib.py}
           {../constr2005/impropers/impropers.py}


\subsection{Sidechain Dihedral Angle $\chi_1$}

The first question asked was "What is the impact of resolution on the distributions of residue $\chi_1$?". The answer
was obtained by constructing and inspecting $p(\chi_1 / R, \mbox{resolution})$ with

\IncludeOut{../constr2005/chi1/impact-of-resolution/make-mdt.py}
           {../constr2005/chi1/impact-of-resolution/mdt.mdt}

and 

\IncludeOut{../constr2005/chi1/impact-of-resolution/asgl.py}
           {../constr2005/chi1/impact-of-resolution/asgl2-a.pdf}

\noindent which clearly show that x-ray structures at resolution of at least 2.0 \Ang\ are just fine. X-ray 
structures above that resolution and NMR structures (whose resolution is set artificially to 
0.45 \Ang\ for the purposes of \MDT\ tabulation) do not appear to be suitable for deriving restraints
for modeling, as the peaks are significantly wider and there is a significant population at ~120\degr. 
Also, the peaks appear Gaussian. Thus, a weighted sum of three Gaussians (except for Pro, which has two) 
was judged to be an appropriate model for these data:

\IncludeOut{../constr2005/chi1/make-mdt.py}
           {../constr2005/chi1/mdt.mdt}
\IncludeOut{../constr2005/chi1/asgl.py}
           {../constr2005/chi1/asgl1-a.pdf}

The weights, means, and standard deviations of the Gaussians were obtained by least-squares fitting
with \ASGL\ and are manually added to the \MODELLER\ \MDT\ library:

\IncludeOut{../constr2005/chi1/fit.top}
           {../constr2005/chi1/fit.ps}

\subsection{Sidechain Dihedral Angle $\chi_2$}

The situation is very similar to that for $\chi_1$, except that the shapes of histograms are
not Gaussian in most cases. Therefore, the 1D cubic splines are used to represent the restraints.

\IncludeOut{../constr2005/chi2/make-mdt.py}
           {../constr2005/chi2/mdt.mdt}
\IncludeOut{../constr2005/chi2/asgl.py}
           {../constr2005/chi2/asgl1-a.pdf}
\IncludeOut{../constr2005/chi2/modlib.py}
           {../constr2005/chi2/chi2.py}

The smoothing parameter \K{prior\_weight} of 10 was selected by trial and
error, inspecting the resulting restraints in 
{\tt modlib-a.ps}, also produced by {\tt modlib.py}.


\subsection{Sidechain Dihedral Angle $\chi_3$}

Exactly the same considerations apply as to $\chi_2$. The corresponding files are

\IncludeOut{../constr2005/chi3/make-mdt.py}
           {../constr2005/chi3/mdt.mdt}
\IncludeOut{../constr2005/chi3/asgl.py}
           {../constr2005/chi3/asgl1-a.pdf}
\IncludeOut{../constr2005/chi3/modlib.py}
           {../constr2005/chi3/chi3.py}

The resulting restraints are plotted in
{\tt modlib-a.ps}, also produced by {\tt modlib.py}.


\subsection{Sidechain Dihedral Angle $\chi_4$}

Exactly the same considerations apply as to $\chi_2$ and $\chi_3$. The corresponding files are

\IncludeOut{../constr2005/chi4/make-mdt.py}
           {../constr2005/chi4/mdt.mdt}
\IncludeOut{../constr2005/chi4/asgl.py}
           {../constr2005/chi4/asgl1-a.pdf}
\IncludeOut{../constr2005/chi4/modlib.py}
           {../constr2005/chi4/chi3.py}

The resulting restraints are plotted in
{\tt modlib-a.ps}, also produced by {\tt modlib.py}.


\subsection{Mainchain Dihedral Angle $\Phi$}

Exactly the same considerations apply as to $\chi_2$, $\chi_3$, and $\chi_4$. The corresponding files are

\IncludeOut{../constr2005/phi/make-mdt.py}
           {../constr2005/phi/mdt.mdt}
\IncludeOut{../constr2005/phi/asgl.py}
           {../constr2005/phi/asgl1-a.ps}
\IncludeOut{../constr2005/phi/modlib.py}
           {../constr2005/phi/phi.py}

The resulting restraints are plotted in
{\tt modlib-a.ps}, also produced by {\tt modlib.py}.


\subsection{Mainchain Dihedral Angle $\Psi$}

Exactly the same considerations apply as to $\chi_2$, $\chi_3$, $\chi_4$, and $\Phi$. The corresponding files are

\IncludeOut{../constr2005/psi/make-mdt.py}
           {../constr2005/psi/mdt.mdt}
\IncludeOut{../constr2005/psi/asgl.py}
           {../constr2005/psi/asgl1-a.pdf}
\IncludeOut{../constr2005/psi/modlib.py}
           {../constr2005/psi/psi.py}

The resulting restraints are plotted in
{\tt modlib-a.ps}, also produced by {\tt modlib.py}.


\subsection{Mainchain Dihedral Angle $\omega$}

This dihedral angle is a little different from all others explored thus far because it 
depends more strongly on the type of the subsequent residue than the type of the 
residue whose dihedral angle is studied; that is, the $\omega$ of the residue preceding 
Pro, not the Pro $\omega$, is impacted by the Pro residue. These dependencies are explored 
with \MDT\ tables in directory {constr2005/omega/run1/}. The bottom
line is that we need to use the \MDT\ residue type feature 28, which refers to the type
of the residue that is \K{deltai} away from the residue with the dihedral
angle $\omega$.

The next step is to obtain the $p(\omega/R_{+1})$ distributions with finer sampling of
0.5\degr:

\IncludeOut{../constr2005/omega/make-mdt.py}
           {../constr2005/omega/mdt.mdt}

The distributions are plotted in the raw and logarithmic form with:

\IncludeOut{../constr2005/omega/asgl.py}
           {../constr2005/omega/asgl1-a.pdf}

\IncludeOut{../constr2005/omega/asgl-log.py}
           {../constr2005/omega/asgl2-a.pdf}

Clearly, the peaks are sharp and will best be modeled by Gaussian distributions.  
Similarly to $\chi_1$, two Gaussian distributions are fit to the histograms with

\IncludeOut{../constr2005/omega/fit.top}
           {../constr2005/omega/fit-a.ps}

The means and standard deviations for each residue type are extracted from {\tt fit.log} by 
the {\tt get\_prms.F} program, but they are only used to guess the means of 179.8\degr\ and 0\degr\ 
and standard deviations of 1.5\degr\ and 1.5\degr\ for the two peaks, respectively. The
distribution of $\omega$ dihedral angles in the models calculated with these $\omega$ restraints 
will be checked carefully and the restraint parameters will be adapted as needed.

The weights of the peaks are not determined reliably by least-squares fitting in this case
because the second weight is very close to 0 (in principle, they can even be less than zero). 
Therefore, they are determined separately
by establishing $p(c_{\omega}/R_{+1})$, where $c_{\omega}$ is the class of the $\omega$
dihedral angle (1 or 2, {\em trans} or {\em cis}).

\IncludeOut{../constr2005/omega/weights/make-mdt.py}
           {../constr2005/omega/weights/mdt.mdt}
\IncludeOut{../constr2005/omega/weights/asgl.py}
           {../constr2005/omega/weights/asgl1-a.pdf}

The library {\tt omega.py} is edited manually to replace the means and standard deviations 
with {\tt 179.8  0.0   2.3  2.3}.


\subsection{Mainchain Dihedral Angles $\Phi$ and $\Psi$}

The initial runs in {\tt run1} explored Ramachandran maps extracted from different representative 
sets of structures (eg, clustered by 40\% sequence identity) and stratification by the 
crystallographic residue Biso as well as resolution and residue type. We ended up with
the sample and stratification described above.

The 2D histograms $p(\Phi, \Psi / R)$ are derived with:

\IncludeOut{../constr2005/phipsi/make-mdt.py}
           {../constr2005/phipsi/mdt.mdt}

They are plotted with

\IncludeOut{../constr2005/phipsi/asgl.py}
           {../constr2005/phipsi/asgl1-a.pdf}

The distributions are clearly not 2D Gaussian functions and need to be approximated by 2D cubic splines.
Exploring and visualizing various smoothing options results in the following file:

\IncludeOut{../constr2005/phipsi/modlib.py}
           {../constr2005/phipsi/phipsi.py}

The raw, smooth, and transformed surfaces are visualized and compared best
with Mathematica.

\section{Non-bonded restraints}

A general pairwise distance- and atom-type dependent statistical potential for all atom type pairs 
has been derived by Min-yi Shen (DOPE). Here, we focus on specialized pairwise non-bonded restraints.

\subsection{Mainchain hydrogen bonding restraints}

The idea is to describe them as restraints on the donor and acceptor pairs of atom triplets.
The donor triplet could be $N_i - CA_{i} - C_{i-1}$ and the acceptor triplet could be 
$O_j - C_j - CA_j$, where $i$ and $j$ are residue indices.

As always, here are the aspects that need to be explored and defined:
\begin{itemize}
\item Which dependent features to use? 

The dependent features are clearly the distance, two angles, and three dihedral 
angles of the two triplets of atoms (the donor and acceptor triplets), though I hope that 
some of them can be omitted without too much of a problem. 

We could have the total potential as a sum of terms for each one of the dependent 
features, but then the correlations between them would be lost. We need to find
out which features are most correlated and join those in the higher order
restraints.

Physically, how does the definition of the H atom position on the donor `eliminate' 
the need to consider the two atoms connected to the the donor N (and thus reduce
the number of dependent features)?

\item Which independent features to use? 

The independent features can be divided into those that the dependent features `really'
depend on and those that are there for quality control (\eg, resolution). 
The independent features include the triplet types (donor and acceptor, 
irrespective of the residue type), sequence separation, and X-ray 
structure resolution. It seems best to fix the triplet types to DON and ACC
respectively and let the sequence separation span the negative and positive range.
This way, the triplet types could even be omitted from the MDT table (they don't 
change), just like the resolution (though I don't like the former omission).

\item What is the range and binning of these features?

The first (4.5 \AA) and second atom shell (8 \AA) are important numbers for considering
the range of the dependent distance. In addition, the standard range of H-bonds
is 3.5 \AA.

It is conceivable that looking only at the raw distributions for deciding about 
the dependent and independent features, their bins, and ranges would be misleading.
For example, normalization of the raw frequency with that expected by chance might
eliminate a large number of differences caused by such features as sequence
separation. Thus, it may be possible to use coarser binning in some independent
features. 

The previous point indicates the need to develop smoothing and normalization
early on, so that `final' restraints and not raw frequencies are used in judging
the selection of features, binning, and range. One should be helped by conditional
entropies and Mathematica in this endeavor.

\item How to smooth the raw frequencies?

By adding a uniform distribution with an appropriate weight.

\item How to normalize the raw frequencies?

The problem is that both the `analytical' and `empirical' routes are very difficult:
(i) duplicating the $4 \Pi r^2$ argument here would require considering volume elements 
spanned by distance, dihedral angles, and angles, which is difficult; (ii) it is 
difficult to imagine what pairs of atom triplets in real structures would provide
a good reference. Min-yi probably came to the rescue with an idea to simulate
pairs of triplets inside a sphere of say 23 \AA radius. Pairs of triplets are placed
randomly inside the sphere, no atom-overlap checks are performed, and then the
distribution of the relative orientations is collected. But there is still a problem
with this idea: Because the reference does not depend on sequence separation, the
reference will not `normalize' out the impact of sequence separation in the raw
frequencies, which does not `feel' right.

Here is the beginning of a larger idea, based on playing games with a system we define,
so we know what it is. It is also based on an idea of progression, evolving the system
from a simple version where everything is clear to a more complex and more realistic
system in a series of steps that are hopefully managable. And it is already clear
up front that the idea will be fighting both the thermodynamic assumption for deriving
the statistical potentials and the multi-body problem (because the idea is 
principled and because these are the two main issues in extracting restraints from 
a sample of structures).

So the starting toy system is a polypeptide chain that `feels' energy terms for only
chemical bonds, angles, and dihedral angles, each one of which depends on the atom types and 
the residue type. There are no other interactions, not even non-bonded 
interactions. The chain looks like a random walk (but of course it is 
not). By definition, the `PDB' (\ie, the sample) contains native structures at 
the global energy minimum, each one of which is entirely self-consistent with each other 
(\ie, there is no frustration among the restraints). Clearly, the sample will show that 
the distributions of the bonds, angles, and dihedral angles are delta functions. Therefore, 
we can in a straighforward way determine the means of all restraints, but there is no 
`entropy' in the resulting restraints and/or it is not knowable from the sample. Nevertheless, 
the corresponding pdf's (\ie, delta functions) would allow us to exactly predict the native 
structure of any new sequence.

But wait a second, we just may have tacitly skipped the `normalization' step because the 
final answer was so obvious. Should we in fact formally normalize the delta functions by a distribution
of bonds, angles, and dihedral angles for a random collection of points (of course, we would
get the same delta functions back)? Why a random collection? Because it seems that the
reference distribution should be based on all energy terms but those we are trying to 
extract from the sample. To make these steps clearer, let's consider them only at the 
2nd level of buildup, next. But we should come back here and define exactly the
properties of the random collection (\eg, What is random? Uniformly distributed in 
Cartesian coordinates? What is volume shape and size? These will impact on the 
distance distribution, though not on the angle and dihedral angle distributions if 
the volume is large enough.).

Is this progression very similar to Min-yi's DOPE manuscript, except that partitioning is 
a little different (there, it is from single-body to two-body)? 

Now comes the step to the 2nd level of the buildup. Let's say that the real chains also 
feel the non-bonded Lennard-Jones terms between all atoms separated by more than 3 chemical 
bonds, in addition to the bond, angle, and dihedral angle terms. The systems is now frustrated 
and determining the Lennard-Jones terms is not easy (if we had hard spheres, then we could
just look for the closest distance in a large sample and be done). Why do I feel that
$-k_B T$ times the natural logarithm of the ratio between the PDB sample distribution of 
non-bonded distances and the non-bonded distribution from the sample in the first step is 
a good approximation to the Lennard-Jones energy (not even PMF, but real potential energy)?

In the end, the order of the steps seems important for the final result, while physically it 
should not be. Should we iterate or explore multiple paths for a self-consistent solution? 
It seems estimating the more determining terms (smaller entropy on their own) first makes sense.

Do it for homology-derived restraints, which are not physical, but the statistical framework
above should still apply. They are strong restraints, so they should probably be considered
early in the progression, right after the stereochemistry.

1) for an ideal system in which all variables are independent
(uncoupled), each variable will be found in its minimum/equilibrium
position. no information about force constants will be revealed

(2) introducing an distance restraint into a protein chain will
inevitably include a variable coupled to the existing internal
coordinates into the minimization, i.e. the restraint "bond length" is
not an independent variable.

(3) the addition of the restraints will expose the force constants for
each existing restraint.

Then some comments on your writeup, which is very cool! I like the specific example of the simplest possible example of a "frustrating" restraint (ie, the restraint on l), with equations. Did you try Mathematica to find the solution of the system of three equations? I am sure they can be solved numerically at the very least. There are probably no insights in the specific solution, which must depend, qualitatively even,  on the values of the means and force constants, but who knows. Also, one could write other systems of three equations by picking say distance l, not theta, as the variable to minimize. I wonder if that would give us any advantage?

- i certainly agree completely with your 3 points below. it is interesting that this "gedanken" experiment addresses both the multiple body problem as well as the concern that PDB is not a boltzmann ensemble. this leads me to believe that we are doing something fundamentally correct.

- it seems you can improve the description of the reference state and normalization in the dope paper, based on the discussion we have here. again, i think it would be best to keep the motivation/rationale/execution of normalization as statistical as possible, as opposed to physical. or at least do so at the beginning, and then make the physical connection in the end, if you must, just like you did overall. on the other hand, our discussion here is physical, not statistical, so i am not sure about the comment i just made.

- still for the DOPE paper, specifically: why did we select the reference state for DOPE the way we did (which interactions are on/off ...).

- i wonder if going from 3 to 4 points would again fundamentally change the situation in 3D (as it did when we went from 2 to 3 points), since we would for the first time introduce chirality. it probably does not matter in ways relevant to us here.

- so now to assumptions/approximations/new questions (for you ;-) ): I suppose we won't be able to solve the problem exactly (actually, it would be very good to exactly define the problem we are addressing: How do we extract the most accurate means and force constants from a sample of native states in our "toy" universe?). So we need to start thinking about suitable assumptions/approximations. what follows uses language so unprecise it is irresponsible to use it. but what can i do! ;-) 

    - it may be possible to come up with reasonable approximations if we assume that the force constants for the added "frustrating" restraints (eg, $k_l$) are very weak compared to the other restraints (eg, bonds, angles, dihedrals). 

    - so maybe in our gedanken experiment we can add increasingly weaker restraints and extract them from the native states and reference states corresponding to global minima from the previous step (what we had on the whiteboard in my office). in addition, maybe a "mean field" solution is the best we can do; ie, after all unsolvable minimum-defining equations end up pushing the means and force constants in all directions, we end up with some kind of a gaussian distribution for them and maybe we CAN estimate the mean and standard deviation of that gaussian, although we cannot get the individual values in it.

- so i'd like to ask you here again how come people use E = -kT  ln p(native) / p(reference) ; what exactly is the native, reference, E for which this equation allows one to calculate E. what is the origin of this equation, approximations, ...?

- it still seems good to derive the expression equivalent to E = -kT  ln p(native) / p(reference) explicitly for your simple 3 body system. Let's push that one to its complete/clear/explicit solution to improve our understanding in general. But you will need to make it more complicated by introducing a few different types of points, you need to imagine a PDB native triangle structures for lots of triangles consisting of the minimal number of types of vertices (just enough to make it useful here). then, how do we get all the means and force constants?

\item Overall considerations:

There is a picture that a short-range, residue type-independent H-bonding potential 
will be useful for restraining the specific local geometry and not for selecting
the fold directly. A longer range orientation-dependent two-body term could be 
helpful in discriminating between different folds.

Presumably, the final restraints will not have a relatively small number of 
simple (Gaussian) peaks, thus the new XOR restraint is probably not indicated here.

Develop some knowledge about the problem by doing lots of 1D histograms for the 
individual features in your sample, both dependent and independent ones.
\end{itemize}

\subsection{Orientation-dependent pairwise non-bonded restraints}

\section{Homology-derived restraints}
